% COMMANDS:
% - bigmid: Dynamically sized mid bar.
% - spacerule: add a centered dashed line with space above and below
% - \dbox{#1}: Adds a nicely formatted slightly grey box around #1
% - \begin{dproof} ... \end{dproof}: A nicely formatted proof. Use \qedhere to place qed
% - \ddef{#1}{#2}: Makes a definition (and counts defs). #1 goes inside parens at beginning, #2 is actual def.
% - \begin{dtable}{#1} ... \end{dtable}: Makes a minimalist table. #1 is the alignment, for example: {clrr} would be a 4 column, center left right right table.

% Dynamically sized mid bar.
\newcommand{\bigmid}{\mathrel{\Big|}}

% ---- Nice Color Palette and Notes ----
\definecolor{dblue}{RGB}{98, 140, 190}
\definecolor{dlblue}{RGB}{216, 235, 255}
\definecolor{dgreen}{RGB}{124, 155, 127}
\definecolor{dpink}{RGB}{207, 166, 208}
\definecolor{dyellow}{RGB}{255, 248, 199}
\definecolor{dgray}{RGB}{46, 49, 49}

% TODO
\newcommand{\todo}[1]{\textcolor{red}{TODO: #1}}
\newcommand{\dnote}[1]{\textcolor{dblue}{Dave: #1}}

% URL
\newcommand{\durl}[1]{\textcolor{dblue}{\underline{\url{#1}}}}

% Circled Numbers
\newcommand*\circled[1]{\tikz[baseline=(char.base)]{\node[shape=circle,draw,inner sep=0.7pt] (char) {\footnotesize{#1}};}}
% From: http://tex.stackexchange.com/questions/7032/good-way-to-make-textcircled-numbers

% Under set numbered subset of equation
\newcommand{\numeq}[3]{\underset{\textcolor{#2}{\circled{#1}}}{\textcolor{#2}{#3}}}

% ---- Abbreviations -----
\newcommand{\tc}[2]{\textcolor{#1}{#2}}
\newcommand{\ubr}[1]{\underbrace{#1}}
\newcommand{\uset}[2]{\underset{#1}{#2}}
\newcommand{\eps}{\varepsilon}

% Typical limit:
\newcommand{\nlim}{\underset{n \rightarrow \infty}{\lim}}
\newcommand{\nsum}{\sum_{i = 1}^n}
\newcommand{\nprod}{\prod_{i = 1}^n}

% Add an hrule with some space
\newcommand{\spacerule}{\begin{center}\hdashrule{2cm}{1pt}{1pt}\end{center}}

% Mathcal and Mathbb
\newcommand{\mc}[1]{\mathcal{#1}}
\newcommand{\indic}{\mathbbm{1}}
\newcommand{\bE}{\mathbb{E}}

\newcommand{\ra}{\rightarrow}
\newcommand{\la}{\leftarrow}

% ---- Figures, Boxes, Theorems, Etc. ----

% Basic Image
\newcommand{\img}[2]{
\begin{center}
\includegraphics[scale=#2]{#1}
\end{center}}

% Put a fancy box around things.
\newcommand{\dbox}[1]{
\begin{mdframed}[roundcorner=4pt, backgroundcolor=gray!5]
\vspace{1mm}
{#1}
\end{mdframed}
}

%  --- PROOFS ---

% Inner environment for Proofs
\newmdenv[
  topline=false,
  bottomline=false,
  rightline = false,
  leftmargin=10pt,
  rightmargin=0pt,
  innertopmargin=0pt,
  innerbottommargin=0pt
]{innerproof}

% Proof Command
\newenvironment{dproof}[1][Proof]{\begin{proof}[#1] \text{\vspace{2mm}} \begin{innerproof}}{\end{innerproof}\end{proof}\vspace{4mm}}

% Quick Definition
\newcounter{DaveDefCounter}
\setcounter{DaveDefCounter}{1}

\newcommand{\ddef}[2]
{
\begin{mdframed}[roundcorner=1pt, backgroundcolor=white]
\vspace{1mm}
{\bf Definition \theDaveDefCounter} (#1): {\it #2}
\stepcounter{DaveDefCounter}
\end{mdframed}
}

% Block Quote
\newenvironment{dblockquote}[2]{
\begin{blockquote}
#2
\vspace{-2mm}\hspace{10mm}{#1} \\
\end{blockquote}}

% Algorithm
\newenvironment{dalg}[1]
{\begin{algorithm}\caption{#1}\begin{algorithmic}}
{\end{algorithmic}\end{algorithm}}


% Quick Table
\newenvironment{dtable}[1]
{\begin{figure}[h]
\centering
\begin{tabular}{#1}\toprule}
{\bottomrule
\end{tabular}
\end{figure}}

% For numbering the last of an align*
\newcommand\numberthis{\addtocounter{equation}{1}\tag{\theequation}}

\DeclareMathOperator*{\argmin}{arg\,min}
\DeclareMathOperator*{\argmax}{arg\,max}

\newtheorem{conjecture}{Conjecture}[section]
\newtheorem{remark}{Remark}[section]
\newtheorem{theorem}{Theorem}[section]
\newtheorem{corollary}{Corollary}[theorem]
\newtheorem{lemma}[theorem]{Lemma}